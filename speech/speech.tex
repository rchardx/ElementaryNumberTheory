%author:  Wentai ZHANG(rchardx@gmail.com)
\def\version{20100122-rev8}

\input{ctex4xetex.cfg}
\documentclass{ctexart}

\usepackage[CJKtextspaces]{xeCJK}
% Copyright(C) 2008,2009 Wentai ZHANG
% Default OpenType Font Configuration for xeCJK

\defaultfontfeatures{Mapping=tex-text}

\def\cjkrm{Adobe Song Std}
\def\cjksf{Adobe Kaiti Std}
\def\cjktt{Adobe Fangsong Std}

\def\cjkbd{Adobe Heiti Std}
\def\cjkit{Adobe Kaiti Std}

\setCJKfamilyfont{rm}[BoldFont=\cjkbd,ItalicFont=\cjkit]{\cjkrm}
\setCJKfamilyfont{sf}[BoldFont=\cjkbd,ItalicFont=\cjkit]{\cjksf}
\setCJKfamilyfont{tt}[BoldFont=\cjkbd,ItalicFont=\cjkit]{\cjktt}

\setCJKmainfont[BoldFont=\cjkbd,ItalicFont=\cjkit]{\cjkrm}
\setCJKsansfont[BoldFont=\cjkbd,ItalicFont=\cjkit]{\cjksf}
\setCJKmonofont[BoldFont=\cjkbd,ItalicFont=\cjkit]{\cjktt}

\setmainfont{TeX Gyre Pagella}
\setsansfont{TeX Gyre Adventor}
\setmonofont{TeX Gyre Cursor}
% \setmainfont{Arno Pro}
% \setsansfont{Myriad Pro}
% \setmonofont{Courier Std}


\usepackage{amsmath,amssymb,amsthm,latexsym,mathrsfs}
\usepackage{enumerate}

\newcommand{\roma}[1]{\romannumeral #1}
\newcommand{\Roma}[1]{\expandafter\@slowromancap\romannumeral #1@}

\newcommand{\bbold}[1]{\textbf{#1}}

\newcommand{\aabs}[1]{{ \left| #1 \right| }}
\newcommand{\ffloor}[1]{{ \left\lfloor #1 \right\rfloor }}
\newcommand{\N}{\boldsymbol{N}}
\newcommand{\Z}{\boldsymbol{Z}}

\newtheorem{thrm}{定理}[section]
\newtheorem{prop}{性质}[section]
\newtheorem{exmp}{例题}[section]
\newtheorem{defn}{定义}[section]
\newtheorem{lemm}[thrm]{引理}
\newtheorem{coro}[thrm]{推论}

\renewenvironment{proof}[1][证]{\noindent \textbf{#1.} }{\hfill$\Box$}

\begin{document}
\title{讲稿}
\author{张文泰\\ \texttt{rchardx@gmail.com}}
\date{\version}
\maketitle

\section{整除}

\subsection{自然数、Peano公理}
主要是最小自然数原理以及数学归纳法。

\begin{exmp}
设$n\neq 1$。证明:$(n-1)^2|n^k-1$的充要条件是$(n-1)|k$。
\end{exmp}
\begin{proof}
$n^k-1=[(n-1)+1]^k-1=A(n-1)^2+k(n-1)$,这里$A$是一个整数,又$(n-1)^2|n^k-1$,所以必然有$(n-1)|k$,反过来命题同样成立。得证。
\end{proof}

\subsection{整除和素数}
素数的检验。

\begin{exmp}
证明:形如$4m+1$的素数有无穷多个。
\end{exmp}
\begin{proof}
设$p_1,p_2,\dotsc,p_r$是所有$4m+1$形式的素数,且$p_i=4m_i+1$。我们令$n=p_1p_2\dotsm p_r+1$,则$n\equiv 2\pmod{4}$。那么如果$n$的素因数全都是$4m+3$形式,那么$n\equiv 1\pmod{4}$或$n\equiv 3\pmod{4}$,绝不会$n\equiv 2\pmod{4}$,所以一定存在一个$p=4m+1 | n$,但是这个$p$没有在$p_1,p_2,\dotsc,p_r$中出现,所以矛盾。
\end{proof}

\subsection{帶余除法、辗转相除法}
重点是辗转相除法及其扩展,并介绍相关的Euclid's Algorithm。

使用Euclid辗转相除法,我们还可以求出$ax+by=d$的解。从后面的内容我们知道,这个不定方程有解的充要条件是$(a,b)|d$。我们先求出$ax+by=(a,b)$的解,然后可以得到原方程的解。具体的做法是使用迭代。我们先考虑和原始的辗转相除法相同的过程,当最后$b=0$时,$a'x=a'$的解是$x=1, y=0$。对于$ax+by=(a,b)$,下一层的运算结果是$bx'+(a\mod b)y'=(a,b)$,那么可以解得$x=y',y=x'-\ffloor{\frac{a}{b}}y'$,因为$\ffloor{\frac{a}{b}}\times b+(a\mod b)=a$。

\begin{exmp}
设$a>2$是奇数。证明
\begin{enumerate}[(i)]
\item 一定存在正整数$d \leq a-1$,使得$a|2^d-1$。
\item 设$d_0$是满足上面的最小正整数$d$。那么,$a|2^h-1$的充要条件是$d_0|h$。
\end{enumerate}
\end{exmp}
\begin{proof}
先证(i)。考虑以下$a$个数:
\begin{displaymath}
2^0,2^1,2^2,\dotsc,2^{a-1}
\end{displaymath}
显然,$a \nmid 2^j(0 \leq j <a )$。由定理知,对于每一个$j$,$0 \leq j <a$,
\begin{displaymath}
2^j=q_j a+r_j, \quad 0<r_j<a
\end{displaymath}
所以$a$个余数$r_0,r_1,\dotsc,r_{a-1}$仅可能取$a-1$个值。因此其中必有两个相等,设为$r_i,r_k$,不妨设$0\leq i<k<a$。因而有
\begin{displaymath}
a(q_k-q_i)=2^k-2^i=2^i(2^{k-i}-1)
\end{displaymath}
所以,$a|2^{k-i}-1$,则$d=k-i\leq a-1$,(i)得证。

下面证(ii)。易证充分性,所以只要证必然性。
\begin{displaymath}
h=qd_0+r, \quad 0\leq r<d_0
\end{displaymath}
因而有
\begin{displaymath}
2^h-1=2^{qd_0+r}-2^r+2^r-1=2^r(2^{qd_0}-1)+(2^r-1)
\end{displaymath}
易得$a|2^r-1$,由此及$d_0$的最小性得$r=0$,即$d_0|h$。
\end{proof}

\begin{exmp}
给定$a, d, n, m$,求$\sum_{i=0}^{n-1}\ffloor{\frac{a+di}{m}}$。
\end{exmp}


\subsection{最大公约数}
几个证明和性质。整系数线性组合。

第一个途径:
\begin{thrm}
$a_j|c(1\leq j\leq k)$的充要条件是$[a_1,\dotsc,a_k]|c$。
\end{thrm}
\begin{proof}
充分性显然。设$L=[a_1,\dotsc,a_k]$。得
\begin{displaymath}
c=qL+r,\quad 0\leq r<L
\end{displaymath}
由此及$a_j|c$推出$a_j|r$,所以$r$是公倍数。进而,又由$0\leq r<L$得$r=0$,所以$L|c$。
\end{proof}

\begin{exmp}
设$p$是素数。证明:若$(a,p)=1$,则$p|a^{p-1}-1$。
\end{exmp}
\begin{proof}
首先要说明组合数的一个性质
\begin{displaymath}
\binom{p}{j}=\frac{p!}{j!(p-j)!}
\end{displaymath}
是整数,即$j!(p-j)!|p!$。由于$p$是素数,所以,对任意$1\leq i\leq p-1$有$(p,i)=1$。因此
\begin{displaymath}
(p,j!(p-j)!)=1,\quad 1\leq j\leq p-1
\end{displaymath}
进而推出,当$1\leq j\leq p-1$时$j!(p-j)!|(p-1)!$,也就是$p|\binom{p}{j}$。

我们先证$p|a^p-a$。用归纳法。当$a=1$时显然成立,若$a=n$时成立,$(n+1)^p-(n+1)=n^p-n+pA$,这里$A$是一个整数。显然,$a=n+1$时也成立,所以$p|a^p-a$。再由$(a,p)=1$,命题得证。
\end{proof}

第二个途径:
\begin{thrm}
设$a_1,\dotsc,a_k$是不全为零的整数。我们有
\begin{enumerate}[(i)]
\item $(a_1,\dotsc,a_k)=\min\{s=a_1x_1+\dotsb+a_k x_k : x_j\in \Z(1\leq j\leq k),s>0\}$,即$a_1,\dotsc,a_k$的最大公约数等于$a_1,\dotsc,a_k$的所有整系数线性组合组成的集合$S$中最小的整数。
\item 一定存在一组整数$x_{1,0},\dotsc,x_{k,0}$使得
\begin{equation}
(a_1,\dotsc,a_k)=a_1 x_{1,0}+\dotsb+a_k x_{k,0}
\end{equation}
\end{enumerate}
\end{thrm}
\begin{proof}
由于$0<a_1^2 +\dotsb+a_k^2\in S$,所以集合$S$中有正整数。由最小自然数原理得$S$中必有最小数$s_0$。显然,对于任一公约数$d|a_j(1\leq j\leq k)$,则必有$d$整除$S$任一元素,那么$d|s_0$。所以,$\aabs{d}\leq s_0$。另外
\begin{displaymath}
a_j=q_j s_0+r_j,\quad 0\leq r_j<s_0
\end{displaymath}
因为$s_0$可以被$x_1,\dotsc,x_k$表示,$a_j$也可以,而$r_j=a_j-q_j s_0$,所以$r_j\in S$。如果$r_j>0$,则与$s_0$最小矛盾,所以$r_j=0$。即$s_0$是最大公约数。
\end{proof}

\begin{exmp}
设$m, n$是正整数。证明$(2^m-1,2^n-1)=2^{(m,n)}-1$。
\end{exmp}
\begin{proof}
不妨设$m \geq n$。由带余除法得$m=qn+r, 0 \leq r < n$。我们有
\begin{displaymath}
2^m-1=2^{qn+r}-2^r+2^r-1=2^r(2^{qn}-1)+2^r-1
\end{displaymath}
由此及$2^n-1|2^{qn}-1$得
\begin{displaymath}
(2^m-1,2^n-1)=(2^n-1,2^r-1)
\end{displaymath}
注意到$(m,n)=(n,r)$,若$r=0$,则$(m,n)=n$,结论成立。若$r>0$,则继续对$(2^n-1,2^r-1)$做同样的处理,由辗转相除法可以知道结论成立。

值得注意的是,这里的2用任何一个大于1的自然数替代都是成立的。

(联想:$(F_m,F_n)=F_{(m,n)}$。)
\end{proof}

\begin{exmp}
设$p$是奇素数,$q$是$2^p-1$的素因数。证明$q=2kp+1$。
\end{exmp}
\begin{proof}
首先有$q|2^{q-1}-1$。再得$q|2^{(p,q-1)}-1$,所以$p|q-1$,再由$p$为奇素数得$q=2kp+1$。
\end{proof}

\begin{exmp}
$13|a^2-7b^2$的充要条件是$13|a, 13|b$。
\end{exmp}
\begin{proof}
充分性显然,证必要性。若$13\nmid a$,那么$13\nmid b$,则一定有$x, y$使得$13x+by=1$。由此及$13|y^2(a^2-7b^2)=(ay)^2-7(by)^2$,($(by)^2=(13x)^2+1-26x$)得到$13|(ay)^2-7$。这个式子不可能被13整除,矛盾。所以$13|a, 13|b$。
\end{proof}


\subsection{算术基本定理}
重点在于分解式。

\begin{exmp}
给你一个正整数$n(n \leq 9\times 10^{14})$,求有多少种方式使得$n$能表示为若干个连续正整数的和。
\end{exmp}

\begin{exmp}
练习题Factor(练习题之后讨论)。
\end{exmp}


\section{同余}

\subsection{同余、同余类、剩余系}
简单地给出Euler函数$\varphi(p^k)$的公式。

\begin{exmp}
给定$k, n$,求$\sum_{i=1}^n(k \mod i)$。
\end{exmp}
\begin{proof}
对于当前的$i$,令$p=k \mod i, q=\ffloor{\frac{k}{i}}$;如果$\ffloor{\frac{k}{i+1}}$的值不变,$k \mod (i+1)$的值必然比当前的余数小$q$,所以可以把$p$一直减$q$直到小于$0$为止(减$\ffloor{\frac{p}{q}}$次),将所得的和加入结果中。同时$i$增加$\ffloor{\frac{p}{q}}$。
\end{proof}

\begin{exmp}
当正整数$m$满足什么条件时$1^3+2^3+\dotsb+(m-1)^3+m^3\equiv 0\pmod{m}$一定成立。
\end{exmp}
\begin{proof}
因为$k^3+(m-k)^3\equiv 0\pmod{m}$,所以$1^3+2^3+\dotsb+(m-1)^3+m^3\equiv 0\equiv m^3+(1^3+(m-1)^3)+\dotsb \pmod{m}$。

当$2\nmid m$,此式成立;当$2|m$时,若$4|m$,仍然成立。综上当$2\nmid m$或$4|m$成立;当$2|m$且$4\nmid m$时不成立。
\end{proof}

\subsection{Euler函数与Fermat-Euler定理}
定理的证明:既约剩余系乘积相同。

\begin{exmp}
给出$n, m$,求$x_1,\dotsc,x_n$的数量,这里$0< x_i \leq m$,且$(x_1,\dotsc,x_n,m)=1$。
\end{exmp}
\begin{proof}
我们考察扩展的欧拉函数$\varphi(m,n)$。讨论。
\end{proof}

\begin{exmp}
练习题Sum(练习题之后讨论)。
\end{exmp}

\subsection{Wilson定理}
\begin{thrm}[Wilson]
设$p$是素数,$r_1,\dotsc,r_{p-1}$是模$p$的既约剩余系,我们有
\begin{displaymath}
r_1\dotsm r_{p-1}\equiv -1\pmod{p}
\end{displaymath}
特别的
\begin{displaymath}
(p-1)!\equiv -1\pmod{p}
\end{displaymath}
\end{thrm}
\begin{proof}
当$p=2$时,结论成立。所以设$p\geq 3$。

对这一组给定的既约剩余系中每一个$r_i$,必然存在一个唯一的$r_j$使得
\begin{equation}\label{EQU:wilson_proof}
r_i r_j\equiv 1\pmod{p}
\end{equation}
使$r_i=r_j$的充要条件是
\begin{displaymath}
r_i^2\equiv 1\pmod{p}
\end{displaymath}
即
\begin{displaymath}
(r_i-1)(r_i+1)\equiv 0\pmod{p}
\end{displaymath}
由于$p$是素数且$p\geq 3$,所以上式成立当且仅当
\begin{displaymath}
r_i-1\equiv 0\pmod{p}
\end{displaymath}
或
\begin{displaymath}
r_i+1\equiv 0\pmod{p}
\end{displaymath}
由于$p\geq 3$,所以这两个条件不能同时成立。因此,在既约剩余系中,除了
\begin{displaymath}
r_i\equiv 1,-1\pmod{p}
\end{displaymath}
这两个数以外,对其它的$r_i$必有$r_j\neq r_i$使得式\ref{EQU:wilson_proof}成立。不妨设$r_1\equiv 1\pmod{p}$,$r_{p-1}\equiv -1\pmod{p}$。这样,在这组剩余系中除去满足上式的两个数以外,其它的数恰好可以按照式\ref{EQU:wilson_proof}两两分完,即满足
\begin{displaymath}
r_2\dotsm r_{p-2}\equiv 1\pmod{p}
\end{displaymath}
所以定理得证。
\end{proof}

\begin{exmp}
设$p$是奇素数,证明
\begin{displaymath}
1^2\cdot 3^2\dotsm (p-2)^2\equiv (-1)^{(p+1)/2}\pmod{p}
\end{displaymath}
\end{exmp}
\begin{proof}
注意到当$p$是奇素数时
\begin{displaymath}
\begin{split}
(p-1)!&=(1\cdot(p-1))(3\cdot(p-3))\dotsm((p-2)(p-(p-2)))\\
      &\equiv(-1)^{(p-1)/2}\cdot 1^2 \cdot 3^2\dotsm (p-2)^2\pmod{p}
\end{split}
\end{displaymath}
由此可以得证。
\end{proof}

\begin{exmp}
设素数$p>5$。证明$(p-1)!+1$不可能是素数的方幂。
\end{exmp}
\begin{proof}
显然,$p|(p-1)!+1$,那么如果$(p-1)!+1$为某个素数的方幂,那么这个素数一定是$p$。注意到$p-1|(p-2)!$,我们设$(p-1)!+1=p^k$,得到$p-1|k$(数学归纳法)。然后可以导出矛盾。

数学归纳法:$p^n-1=(p-1+1)^n-1=A(p-1)^2+n(p-1)$。

矛盾:$(p-1)!+1=p^k \leq p^(p-1)$,而左边明显小于右边。
\end{proof}

\begin{exmp}
练习题Color(练习题做完后讨论)。
\end{exmp}

\section{同余方程}
注意多项式同余的最高“有效”次数。

\subsection{一次同余方程}

\subsection{一次同余方程组、孙子定理}

\begin{exmp}
设$k$是给定的正整数。证明:一定存在$k$个相邻整数,其中任何一个数都能被大于$1$的立方数整除。
\end{exmp}
\begin{proof}
设$p_1,\dotsc,p_k$,考虑$x\equiv -j+1\pmod{p_j^3}, j=1,\dotsc,k$。若$x_0$是一个解,那么$x_0,x_0+1,\dotsc,x_0+k-1$就是答案。
\end{proof}

\subsection{模为素数的二次同余方程}

\begin{thrm}[Euler判别法]
设素数$p>2$,$p\nmid d$。那么,$d$是模$p$的二次剩余的充要条件是
\begin{equation}\label{EQU:euler_judge}
d^{(p-1)/2}\equiv 1\pmod{p}
\end{equation}
$d$是模$p$的二次非剩余的充要条件是
\begin{equation}
d^{(p-1)/2}\equiv -1\pmod{p}
\end{equation}
\end{thrm}
\begin{proof}
我们先证明上述两个式子有且仅有一个成立。我们有
\begin{displaymath}
d^{p-1}\equiv 1\pmod{p}
\end{displaymath}
所以
\begin{displaymath}
(d^{(p-1)/2}-1)(d^{(p-1)/2}+1)\equiv 0\pmod{p}
\end{displaymath}
由于素数$p>2$以及
\begin{displaymath}
(d^{(p-1)/2}-1,d^{(p-1)/2}+1)|2
\end{displaymath}
所以,上面两个判别式有且仅有一个成立。

下面证式\ref{EQU:euler_judge}成立是$d$为$p$的二次剩余的充要条件。先证必要性。若$d$是$p$的二次剩余,则必有$x_0$使得
\begin{displaymath}
x_0^2\equiv d\pmod{p}
\end{displaymath}
因而
\begin{displaymath}
x_0^{p-1}\equiv d^{(p-1)/2}\pmod{p}
\end{displaymath}
由于$p\nmid d$,所以$p|x_0$,因而
\begin{displaymath}
x_0^{p-1}\equiv 1\pmod{p}
\end{displaymath}
必要性得证。

再证充分性。证明方法和我们证明Wilson定理的方法一样。设式\ref{EQU:euler_judge}成立,此时必然有$p\nmid d$。考虑
\begin{displaymath}
ax\equiv d\pmod{p}
\end{displaymath}
对于$1\leq i<j\leq (p-1)/2$,我们有
\begin{displaymath}
i^2\not\equiv j^2\pmod{p}
\end{displaymath}
所以,$p$的既约剩余系中两两可以搭配成上式的解,所以
\begin{displaymath}
(p-1)!\equiv d^{(p-1)/2}\equiv -1\pmod{p}
\end{displaymath}
矛盾,充分性得证。
\end{proof}

\begin{exmp}
对于给定的$n$,
\begin{itemize}
\item 求出$x^2\equiv 1\pmod{n}$的解数;
\item 求出所有$x$满足$x^2\equiv 1\pmod{n}$。
\end{itemize}
\end{exmp}
\begin{proof}
我们先假定$n$的不同素因数个数是$r$。考察$x^2\equiv 1\pmod{p^k}$,我们发现由于$(x-1)(x+1)\equiv 1\pmod{p^k}$,所以当$p>2$时只能完全分配到一边,就是两个解。当$p=2$时,由于$(x-1)$和$(x+1)$只相差$2$,只能同时被$2$整除。我们令符号$[p]$表示命题$p$成立时值为$1$,否则为$0$。那么$p=2$时的解数就是$2^{[8|p^k]+[4|p^k]-[2|p^k]}$。我们考虑到各个不同的同余方程是独立的,不同的素因数之间可以分开计算,所以总的解数是$2^{r+[8|n]+[4|n]-[2|n]}$。
\end{proof}

\subsection{Legendre符号、Gauss二次互反律}

\section{不定方程}

\subsection{Pythagoras方程}

\begin{exmp}
不定方程
\begin{displaymath}
x^4+y^4=z^2
\end{displaymath}
无$xyz\neq 0$的解。
\end{exmp}
\begin{proof}
该命题即要证无正整数解。假若有正整数解,那么在全体正整数解中,必有一组解$x_0,y_0,z_0$,使得$z_0$取最小值。
\begin{enumerate}[(i)]
\item 必有$(x_0,y_0)=1$。若不然,有素数$p|x_0,p|y_0$,则推出$p^2|z_0$,与$z_0$的最小性矛盾。由$x_0^2,y_0^2,z_0$为方程$x^2+y^2=z^2$的本原解得,$x_0,y_0$必为一奇一偶,不妨设$2|y_0$,以及$(z_0,y_0)=1$。
\item $g_1=(z_0-y_0^2,z_0+y_0^2)=1$。因为$g_1|(2z_0,2y_0^2)=2(z_0,y_0^2)=2$,由此及$2\nmid z_0-y_0^2$即得$g_1=1$。由此及
\begin{displaymath}
(z_0-y_0^2)(z_0+y_0^2)=x_0^4
\end{displaymath}
我们令
\begin{displaymath}
z_0-y_0^2=u^4,\quad z_0+y_0^2=v^4
\end{displaymath}
这里$v>u>0,(u,v)=1,2\nmid uv$。进而有
\begin{equation}
y_0^2=(v^2-u^2)\frac{v^2+u^2}{2}
\end{equation}
\item $g_2=(v^2-u^2,(v^2+u^2)/2)=1$。因为
\begin{displaymath}
g_2|(v^2-u^2,v^2+u^2)|(2v^2,2u^2)=2(v^2,u^2)=2
\end{displaymath}
由$2\nmid uv$得到$2\nmid (v^2+u^2)/2$,因此$g_2=1$。我们再令
\begin{displaymath}
v^2-u^2=a^2,\quad (v^2+u^2)/2=b^2
\end{displaymath}
这里$a>0,b>0,(a,b)=1$,及$2|a,2\nmid b$。
\item 由$u,v$满足的条件及$a,b$得
\begin{displaymath}
0<b<v<z_0
\end{displaymath}
及$u,a,v$是方程$x^2+y^2=z^2$的本原解且$2|a$。因此得到
\begin{displaymath}
u=r^2-s^2,\quad a=2rs,\quad v=r^2+s^2
\end{displaymath}
所以
\begin{displaymath}
r^4+s^4=b^2
\end{displaymath}
而$b<z_0$,与$z_0$的最小性矛盾,命题得证。
\end{enumerate}
\end{proof}

\subsection{Lagrange定理}

每个正整数一定可以表示为四个平方数之和,即对任意的$n\geq 1$,不定方程
\begin{displaymath}
x_1^2+x_2^2+x_3^2+x_4^2=n
\end{displaymath}
有解。

如果需要证明这个定理,下面的恒等式是必须引进的:
\begin{eqnarray*}
&&(a_1^2+a_2^2+a_3^2+a_4^2)(b_1^2+b_2^2+b_3^2+b_4^2)\\
&&=(a_1b_1+a_2b_2+a_3b_3+a_4b_4)^2+(a_1b_2-a_2b_1+a_3b_4-a_4b_3)^2\\
&&+(a_1b_3-a_3b_1+a_4b_2-a_2b_4)^2+(a_1b_4-a_4b_1+a_2b_3-a_3b_2)^2
\end{eqnarray*}

\end{document}
